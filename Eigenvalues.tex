\documentclass[12pt]{article}
\usepackage[utf8]{vietnam}
\usepackage{amsmath, amssymb}
\usepackage{geometry}
\usepackage{hyperref}
\geometry{a4paper, margin=2.5cm}

\title{Giá trị riêng, vector riêng và chéo hóa ma trận}
\author{}
\date{}

\begin{document}

\maketitle

\section*{1. Giá trị riêng và vector riêng}

Cho một ma trận vuông \( A \in \mathbb{R}^{n \times n} \). Một số thực (hoặc phức) \( \lambda \) được gọi là \textbf{giá trị riêng} (eigenvalue) của \( A \) nếu tồn tại vector không-zero \( \vec{v} \in \mathbb{R}^n \) sao cho:
\[
A\vec{v} = \lambda \vec{v}
\]
Vector \( \vec{v} \) được gọi là \textbf{vector riêng} (eigenvector) tương ứng với giá trị riêng \( \lambda \).

\subsection*{Tìm giá trị riêng}

Để tìm giá trị riêng, giải phương trình:
\[
\det(A - \lambda I) = 0
\]
Đây được gọi là phương trình đặc trưng (characteristic equation) của ma trận \( A \).

\subsection*{Tìm vector riêng}

Sau khi tìm được các giá trị riêng \( \lambda \), thay vào:
\[
(A - \lambda I)\vec{v} = \vec{0}
\]
và giải hệ để tìm vector \( \vec{v} \neq \vec{0} \).

\section*{2. Chéo hóa ma trận}

Một ma trận \( A \in \mathbb{R}^{n \times n} \) được gọi là \textbf{chéo hóa được} nếu tồn tại ma trận khả nghịch \( P \) sao cho:
\[
A = PDP^{-1}
\]
trong đó \( D \) là ma trận đường chéo chứa các giá trị riêng của \( A \), và các cột của \( P \) là các vector riêng tương ứng.

\textbf{Điều kiện chéo hóa:} Ma trận \( A \) chéo hóa được khi và chỉ khi nó có đủ \( n \) vector riêng tuyến tính độc lập.

\section*{3. Ví dụ minh họa}

Cho ma trận:
\[
A = \begin{bmatrix}
4 & 1 \\
2 & 3
\end{bmatrix}
\]

\subsection*{Bước 1: Tìm giá trị riêng}

Tìm nghiệm của phương trình:
\[
\det(A - \lambda I) = 0
\Rightarrow \det \begin{bmatrix}
4 - \lambda & 1 \\
2 & 3 - \lambda
\end{bmatrix} = 0
\]
\[
(4 - \lambda)(3 - \lambda) - 2 = \lambda^2 - 7\lambda + 10 = 0
\Rightarrow \lambda_1 = 5,\quad \lambda_2 = 2
\]

\subsection*{Bước 2: Tìm vector riêng}

\underline{Với \( \lambda = 5 \)}:
\[
(A - 5I)\vec{v} = 0 \Rightarrow 
\begin{bmatrix}
-1 & 1 \\
2 & -2
\end{bmatrix}
\vec{v} = 0 \Rightarrow v_1 = v_2
\Rightarrow \vec{v}_1 = \begin{bmatrix}1 \\ 1\end{bmatrix}
\]

\underline{Với \( \lambda = 2 \)}:
\[
(A - 2I)\vec{v} = 0 \Rightarrow 
\begin{bmatrix}
2 & 1 \\
2 & 1
\end{bmatrix}
\vec{v} = 0 \Rightarrow 2v_1 + v_2 = 0 \Rightarrow v_2 = -2v_1
\Rightarrow \vec{v}_2 = \begin{bmatrix}1 \\ -2\end{bmatrix}
\]

\subsection*{Bước 3: Chéo hóa}

Tạo ma trận:
\[
P = \begin{bmatrix}
1 & 1 \\
1 & -2
\end{bmatrix}, \quad 
D = \begin{bmatrix}
5 & 0 \\
0 & 2
\end{bmatrix}
\Rightarrow A = PDP^{-1}
\]

\section*{4. Kết luận}

- Giá trị riêng cho biết mức độ co giãn của không gian theo các hướng riêng.
- Vector riêng là những hướng không đổi khi tác động bởi phép biến đổi tuyến tính \( A \).
- Chéo hóa giúp đơn giản hóa tính toán lũy thừa ma trận và giải hệ phương trình vi phân tuyến tính.

\href{tratoa.github.io/MAE101/Eigenvalues.html}{a}

\end{document}
