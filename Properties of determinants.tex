\documentclass[a4paper,12pt]{article}
\usepackage[utf8]{vietnam}
\usepackage[T5]{fontenc}
\usepackage{amsmath, amsfonts}
\usepackage{geometry}
\geometry{margin=1in}

\title{\textbf{Tính chất của định thức và ví dụ minh hoạ}}
\author{}
\date{}

\begin{document}
    \maketitle

    \section*{1. Tính chất của định thức}
    \begin{enumerate}
        \item Định thức của ma trận tam giác (trên hoặc dưới) bằng tích các phần tử trên đường chéo chính.
        \item Định thức của ma trận chuyển vị bằng định thức của ma trận ban đầu: 
        \[
        \operatorname{det}(A^T) = \operatorname{det}(A)
        \]
        \item Nếu một hàng hoặc một cột của ma trận toàn số 0 thì định thức bằng 0.
        \item Nếu hai hàng hoặc hai cột của ma trận giống nhau thì định thức bằng 0.
        \item Hoán đổi hai hàng (hoặc hai cột) thì định thức đổi dấu.
        \item Nhân một hàng (hoặc một cột) với một số vô hướng \( k \) thì định thức bị nhân với \( k \).
        \item Nếu nhân toàn bộ ma trận \( A \) cấp \( n \) với \( k \) thì: 
        \[
        \operatorname{det}(kA) = k^n \cdot \operatorname{det}(A)
        \]
        \item Cộng bội của một hàng vào hàng khác thì định thức không đổi.
        \item Định thức của ma trận đơn vị là 1.
        \item Ma trận khả nghịch khi và chỉ khi định thức khác 0.
        \item Định thức của tích hai ma trận bằng tích các định thức:
        \[
        \operatorname{det}(AB) = \operatorname{det}(A) \cdot \operatorname{det}(B)
        \]
        \item Định thức của ma trận nghịch đảo:
        \[
        \operatorname{det}(A^{-1}) = \dfrac{1}{\operatorname{det}(A)}
        \]
        \item Nếu thay đổi thứ tự các cột hoặc hàng theo một hoán vị lẻ thì định thức đổi dấu.
        \item Định thức của ma trận có hai hàng tỉ lệ (song song) bằng 0.
        \item Nếu ma trận có một hàng là tổ hợp tuyến tính của các hàng còn lại thì định thức bằng 0.
        \item Định thức tuyến tính theo từng hàng (hoặc từng cột).
    \end{enumerate}

    \section*{2. Ví dụ minh hoạ}
    Cho:
    \[
    \operatorname{det} \begin{bmatrix}
        a & b & c \\
        p & q & r \\
        x & y & z
    \end{bmatrix} = 3
    \]

    Tính:
    \[
    \operatorname{det} \begin{bmatrix}
        a + 2x & b + 2y & c + 2z \\
        3a - p & 3b - q & 3c - r \\
        a & b & c
    \end{bmatrix}
    \]

    \subsection*{Lời giải:}

    Áp dụng các phép biến đổi sơ cấp dòng (không làm thay đổi hoặc thay đổi có kiểm soát định thức):

    \begin{align*}
        R_1 &\leftarrow R_1 - 2R_3 \quad \text{(định thức không đổi)} \\
        R_2 &\leftarrow R_2 - 3R_3 \quad \text{(định thức không đổi)} \\
        R_1 &\leftarrow -R_1,\quad R_2 \leftarrow -R_2 \quad \text{(đổi dấu 2 lần → định thức giữ nguyên)} \\
        \Rightarrow \text{Ma trận trở thành: } &
        \begin{bmatrix}
            a - 2x & b - 2y & c - 2z \\
            p & q & r \\
            a & b & c
        \end{bmatrix} \\
        R_3 &\leftarrow R_3 - R_1 \Rightarrow
        \begin{bmatrix}
            a - 2x & b - 2y & c - 2z \\
            p & q & r \\
            2x & 2y & 2z
        \end{bmatrix} \\
        \Rightarrow \text{Tách hệ số 2 ra khỏi hàng 3: } &
        2 \cdot \operatorname{det} \begin{bmatrix}
            a - 2x & b - 2y & c - 2z \\
            p & q & r \\
            x & y & z
        \end{bmatrix} \\
        R_1 &\leftarrow R_1 + 2R_3 \Rightarrow
        2 \cdot \operatorname{det} \begin{bmatrix}
            a & b & c \\
            p & q & r \\
            x & y & z
        \end{bmatrix} = 2 \cdot 3 = 6
    \end{align*}

    \textbf{Kết luận:} Giá trị định thức cần tìm là 6.

\end{document}
