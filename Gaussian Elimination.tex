\documentclass[12pt]{article}
\usepackage[utf8]{vietnam}
\usepackage{amsmath}
\usepackage[margin=2cm]{geometry}
\usepackage{array}

\begin{document}

\section*{Phép khử Gauss}

\textbf{Định nghĩa:} Phép khử Gauss là một thuật toán biến đổi ma trận về dạng bậc thang dòng bằng cách sử dụng các phép biến đổi sơ cấp trên dòng. Đây là một công cụ quan trọng để:

\begin{itemize}
    \item Giải hệ phương trình tuyến tính.
    \item Tính hạng của ma trận.
    \item Tính định thức.
    \item Tìm ma trận nghịch đảo.
\end{itemize}

\textbf{Ba phép biến đổi sơ cấp trên dòng:}
\begin{enumerate}
    \item Đổi chỗ hai dòng: \( R_i \leftrightarrow R_j \)
    \item Nhân một dòng với một số khác 0: \( R_i \leftarrow k \cdot R_i \), với \( k \ne 0 \)
    \item Cộng bội của dòng này vào dòng khác: \( R_i \leftarrow R_i + k \cdot R_j \)
\end{enumerate}

\section*{Ví dụ:}
Giải hệ phương trình sau bằng phép khử Gauss:

\[
\begin{cases}
x + y + z = 6 \\
2x + 3y + 7z = 20 \\
x + 3y + 4z = 14
\end{cases}
\]

Ma trận mở rộng tương ứng là:

\[
\begin{bmatrix}
1 & 1 & 1 & 6 \\
2 & 3 & 7 & 20 \\
1 & 3 & 4 & 14
\end{bmatrix}
\]

\textbf{Bước 1:} \( R_2 \leftarrow R_2 - 2R_1 \), \( R_3 \leftarrow R_3 - R_1 \)
\[
\begin{bmatrix}
1 & 1 & 1 & 6 \\
0 & 1 & 5 & 8 \\
0 & 2 & 3 & 8
\end{bmatrix}
\]

\textbf{Bước 2:} \( R_3 \leftarrow R_3 - 2R_2 \)
\[
\begin{bmatrix}
1 & 1 & 1 & 6 \\
0 & 1 & 5 & 8 \\
0 & 0 & -7 & -8
\end{bmatrix}
\]

\textbf{Bước 3:} \( R_3 \leftarrow -\frac{1}{7}R_3 \)
\[
\begin{bmatrix}
1 & 1 & 1 & 6 \\
0 & 1 & 5 & 8 \\
0 & 0 & 1 & \frac{8}{7}
\end{bmatrix}
\]

\textbf{Bước 4:} Khử ngược:
\begin{align*}
R_2 &\leftarrow R_2 - 5R_3 \\
R_1 &\leftarrow R_1 - R_3
\end{align*}

\[
\begin{bmatrix}
1 & 1 & 0 & \frac{34}{7} \\
0 & 1 & 0 & \frac{4}{7} \\
0 & 0 & 1 & \frac{8}{7}
\end{bmatrix}
\]

\textbf{Bước 5:} \( R_1 \leftarrow R_1 - R_2 \)
\[
\begin{bmatrix}
1 & 0 & 0 & \frac{30}{7} \\
0 & 1 & 0 & \frac{4}{7} \\
0 & 0 & 1 & \frac{8}{7}
\end{bmatrix}
\]

\textbf{Kết luận:} Nghiệm của hệ phương trình là:
\[
x = \frac{30}{7}, \quad y = \frac{4}{7}, \quad z = \frac{8}{7}
\]

\end{document}
