\documentclass[12pt]{article}
\usepackage[utf8]{vietnam}
\usepackage{amsmath, amssymb, geometry}
\geometry{margin=1in}
\setlength{\parskip}{1em}
\setlength{\parindent}{0pt}

\title{Phân tích chi tiết các phép biến đổi sơ cấp và kiểm tra tính độc lập tuyến tính}
\date{}
\begin{document}
\maketitle

\textbf{1. Phép biến đổi sơ cấp trên dòng hoặc cột của ma trận}

Phép biến đổi sơ cấp là những phép toán cơ bản thực hiện trên dòng hoặc cột của một ma trận, bao gồm:

\begin{itemize}
    \item Đổi chỗ hai dòng (hoặc hai cột) với nhau.
    \item Nhân một dòng (hoặc một cột) với một số khác 0.
    \item Cộng (hoặc trừ) một bội của một dòng (hoặc cột) này vào dòng (hoặc cột) khác.
\end{itemize}

Ví dụ:

\[
A = \begin{bmatrix} 1 & 2 \\ 3 & 4 \end{bmatrix}
\]

Thực hiện các phép biến đổi sơ cấp:

\begin{align*}
\text{Nhân dòng 1 với 2:} &\quad \begin{bmatrix} 2 & 4 \\ 3 & 4 \end{bmatrix} \\
\text{Dòng 2 trừ 3 lần dòng 1:} &\quad \begin{bmatrix} 1 & 2 \\ 0 & -2 \end{bmatrix}
\end{align*}

\vspace{1em}
\textbf{2. Phép khử Gauss và khử Gauss–Jordan}

\begin{itemize}
    \item \textbf{Phép khử Gauss} nhằm đưa ma trận về \textit{dạng bậc thang} (row echelon form - REF).
    \item \textbf{Phép khử Gauss–Jordan} nhằm đưa ma trận về \textit{dạng bậc thang rút gọn} (reduced row echelon form - RREF), nơi các phần tử chính là 1 và cả trên lẫn dưới đều là 0.
\end{itemize}

Ví dụ:

\[
A = \begin{bmatrix} 1 & 2 & -1 \\ 2 & 4 & 1 \\ -1 & -2 & 5 \end{bmatrix}
\]

Sau khử Gauss:

\[
\begin{bmatrix} 1 & 2 & -1 \\ 0 & 0 & 3 \\ 0 & 0 & 0 \end{bmatrix}
\]

Sau khử Gauss–Jordan:

\[
\begin{bmatrix} 1 & 2 & 0 \\ 0 & 0 & 1 \\ 0 & 0 & 0 \end{bmatrix}
\]

\vspace{1em}
\textbf{3. Hạng của ma trận}

Sau khi biến đổi về dạng bậc thang, \textbf{số dòng khác 0} của ma trận chính là \textit{hạng} (rank) của ma trận đó.

Ví dụ:

\[
B = \begin{bmatrix} 1 & 2 & 3 \\ 0 & 1 & 2 \\ 0 & 0 & 0 \end{bmatrix}
\Rightarrow \operatorname{rank}(B) = 2
\]

\vspace{1em}
\textbf{4. Kiểm tra tính độc lập tuyến tính của hệ vectơ}

Gọi \( \vec{v}_1, \vec{v}_2, \ldots, \vec{v}_k \in \mathbb{R}^n \). Ta xét ba trường hợp:

\textbf{Trường hợp 1: \( k = n \)}

Lập định thức của ma trận có các vectơ là cột. Nếu:

\[
\det \neq 0 \Rightarrow \text{hệ độc lập tuyến tính}, \quad \det = 0 \Rightarrow \text{hệ phụ thuộc tuyến tính}
\]

Ví dụ:

\[
\vec{v}_1 = \begin{bmatrix} 1 \\ 0 \\ 0 \end{bmatrix},\ 
\vec{v}_2 = \begin{bmatrix} 0 \\ 1 \\ 0 \end{bmatrix},\ 
\vec{v}_3 = \begin{bmatrix} 0 \\ 0 \\ 1 \end{bmatrix}
\]

\[
\det\begin{bmatrix} 1 & 0 & 0 \\ 0 & 1 & 0 \\ 0 & 0 & 1 \end{bmatrix} = 1 \Rightarrow \text{độc lập tuyến tính}
\]

\textbf{Trường hợp 2: \( k > n \)}

Trong không gian \( \mathbb{R}^n \), không thể có quá \( n \) vectơ độc lập tuyến tính. Do đó:

\[
k > n \Rightarrow \text{hệ phụ thuộc tuyến tính}
\]

Ví dụ: 4 vectơ trong \( \mathbb{R}^3 \) chắc chắn phụ thuộc tuyến tính.

\textbf{Trường hợp 3: \( k < n \)}

Lập ma trận có các vectơ là cột. Gọi \( r \) là hạng của ma trận.

\[
\begin{cases}
r = k & \Rightarrow \text{hệ độc lập tuyến tính} \\
r < k & \Rightarrow \text{hệ phụ thuộc tuyến tính}
\end{cases}
\]

Ví dụ:

\[
\vec{v}_1 = \begin{bmatrix} 1 \\ 0 \\ 1 \\ 0 \end{bmatrix},\ 
\vec{v}_2 = \begin{bmatrix} 0 \\ 1 \\ 0 \\ 1 \end{bmatrix}
\Rightarrow 
A = \begin{bmatrix} 1 & 0 \\ 0 & 1 \\ 1 & 0 \\ 0 & 1 \end{bmatrix}
\]

Hạng của \( A \) là 2 = số vectơ → hệ độc lập tuyến tính.

\end{document}
