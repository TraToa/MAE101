\documentclass[12pt,a4paper]{article}
\usepackage[utf8]{vietnam}
\usepackage{amsmath,amsfonts}
\usepackage{hyperref}
\usepackage{geometry}
\geometry{margin=2.5cm}

\title{\textbf{Các Khái Niệm Cơ Bản về Ma Trận}}
\author{}
\date{}

\begin{document}

\maketitle

\section*{1. Minor (Phần phụ)}

Cho ma trận vuông $A = [a_{ij}] \in \mathbb{R}^{n \times n}$. \\
\textbf{Minor} của phần tử $a_{ij}$ là định thức của ma trận con $A_{(i,j)}$ được tạo bằng cách loại bỏ hàng $i$ và cột $j$ khỏi $A$:

\[
M_{ij} = \det(A_{(i,j)})
\]

\section*{2. Cofactor (Phần bù đại số)}

\textbf{Cofactor} của phần tử $a_{ij}$ được định nghĩa bởi:

\[
C_{ij} = (-1)^{i + j} \cdot M_{ij}
\]

Tập hợp tất cả các $C_{ij}$ tạo thành ma trận cofactor.

\section*{3. Determinant (Định thức)}

Định thức của ma trận $A$ được tính bằng khai triển Laplace theo hàng đầu (hoặc hàng bất kỳ):

\[
\det(A) = \sum_{j=1}^{n} a_{1j} \cdot C_{1j} = \sum_{j=1}^{n} a_{1j} \cdot (-1)^{1+j} \cdot \det(A_{(1,j)})
\]

\textbf{Lưu ý:} Định thức chỉ xác định cho ma trận vuông và là điều kiện để ma trận khả nghịch.

\section*{4. Adjugate Matrix (Ma trận phụ hợp)}

Ma trận phụ hợp (adjugate) của $A$ là chuyển vị của ma trận cofactor:

\[
\operatorname{adj}(A) = \left[ C_{ij} \right]^\top
\]

\section*{5. Matrix Inverse (Ma trận nghịch đảo)}

Ma trận $A^{-1}$ tồn tại khi và chỉ khi $\det(A) \neq 0$. Công thức tính:

\[
A^{-1} = \frac{1}{\det(A)} \cdot \operatorname{adj}(A)
\]

\textbf{Điều kiện:} $\det(A) \ne 0$. Nếu không, ma trận không khả nghịch.

\section*{6. Tóm tắt quy trình tính $A^{-1}$}

\begin{enumerate}
    \item Tính $\det(A)$. Nếu $\det(A) = 0$ thì dừng lại.
    \item Với mỗi phần tử $a_{ij}$, tính minor $M_{ij}$ và cofactor $C_{ij}$.
    \item Lập ma trận cofactor $C = [C_{ij}]$.
    \item Tính ma trận phụ hợp $\operatorname{adj}(A) = C^\top$.
    \item Tính ma trận nghịch đảo:

    \[
    A^{-1} = \frac{1}{\det(A)} \cdot \operatorname{adj}(A)
    \]
\end{enumerate}

\end{document}
