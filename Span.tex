\documentclass[12pt]{article}
\usepackage[utf8]{vietnam}
\usepackage{amsmath, amssymb, geometry}
\geometry{margin=1in}
\setlength{\parskip}{1em}
\setlength{\parindent}{0pt}

\title{Tổ hợp tuyến tính, Phụ thuộc tuyến tính, Span và Cơ sở}
\date{}
\begin{document}
\maketitle

\textbf{1. Phụ thuộc tuyến tính}

Một tập các vectơ \( \vec{v}_1, \vec{v}_2, \ldots, \vec{v}_k \) được gọi là \textit{phụ thuộc tuyến tính} nếu tồn tại các hệ số \( \lambda_1, \lambda_2, \ldots, \lambda_k \), không đồng thời bằng 0, sao cho:
\[
\lambda_1 \vec{v}_1 + \lambda_2 \vec{v}_2 + \cdots + \lambda_k \vec{v}_k = \vec{0}
\]

\textbf{Ví dụ:}

\[
\vec{v}_1 = \begin{bmatrix} 1 \\ 2 \\ 3 \end{bmatrix}, \quad \vec{v}_2 = \begin{bmatrix} 2 \\ 4 \\ 6 \end{bmatrix} \Rightarrow \vec{v}_2 = 2\vec{v}_1
\Rightarrow \text{phụ thuộc tuyến tính}
\]

\textbf{2. Tổ hợp tuyến tính}

Tổ hợp tuyến tính của các vectơ \( \vec{v}_1, \vec{v}_2, \ldots, \vec{v}_k \) là biểu thức dạng:
\[
\lambda_1 \vec{v}_1 + \lambda_2 \vec{v}_2 + \cdots + \lambda_k \vec{v}_k, \quad \lambda_i \in \mathbb{R}
\]

\textbf{Ví dụ:}

\[
3 \begin{bmatrix} 1 \\ 0 \end{bmatrix} - 2 \begin{bmatrix} 0 \\ 1 \end{bmatrix} = \begin{bmatrix} 3 \\ -2 \end{bmatrix}
\]

\textbf{3. Span (Không gian con sinh bởi một hệ vectơ)}

\[
\operatorname{span}(\vec{v}_1, \vec{v}_2, \ldots, \vec{v}_k) = \left\{ \sum_{i=1}^k \lambda_i \vec{v}_i \mid \lambda_i \in \mathbb{R} \right\}
\]

\textbf{Ví dụ:}

\[
\vec{v}_1 = \begin{bmatrix} 1 \\ 0 \end{bmatrix},\ 
\vec{v}_2 = \begin{bmatrix} 0 \\ 1 \end{bmatrix} \Rightarrow \operatorname{span}(\vec{v}_1, \vec{v}_2) = \mathbb{R}^2
\]

\textbf{4. Cơ sở của một không gian}

Một tập hợp các vectơ là một \textit{cơ sở} của không gian nếu:
\begin{itemize}
    \item Các vectơ \textbf{độc lập tuyến tính}.
    \item Tập đó \textbf{sinh ra toàn bộ không gian} (span).
\end{itemize}

\textbf{Ví dụ:}

Trong \( \mathbb{R}^3 \), tập:
\[
\left\{ \begin{bmatrix} 1 \\ 0 \\ 0 \end{bmatrix},\ 
\begin{bmatrix} 0 \\ 1 \\ 0 \end{bmatrix},\ 
\begin{bmatrix} 0 \\ 0 \\ 1 \end{bmatrix} \right\}
\]
là cơ sở chuẩn của \( \mathbb{R}^3 \).

\end{document}
